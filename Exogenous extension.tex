% !TeX spellcheck = da_DK
\documentclass[11pt,a4paper,oneside]{article}

% Overordnet opsætning
\setlength\parindent{24pt}
\setlength\parskip{3pt}
\setlength{\headheight}{14pt}
\renewcommand{\baselinestretch}{1.5}

\usepackage[utf8]{inputenc}
\usepackage[danish,english]{babel}
\usepackage{amsfonts,amsmath} %math: align og gather mm.; fonts: flere forskellige symboler
\usepackage[left=2cm, right=2cm, bottom=2cm]{geometry}
\usepackage{enumitem} %anvendes til ændring i itemize mm.
\usepackage{fancyhdr} %Opsaening af sidehoved og -fod
\usepackage{lastpage} % Side "x af X".
\usepackage[ocgcolorlinks,linkcolor=black]{hyperref}
\usepackage{indentfirst} %"Indent" efter section og chapters etc.
\usepackage{graphicx} % for plotting graphs in matrix
\usepackage{subcaption} % figure "footnotes"
\usepackage[font=small,compatibility=false]{caption}
\usepackage[nottoc,numbib]{tocbibind}    
\usepackage{ mathrsfs }
\usepackage{bbold}

% bibliography
\usepackage{csquotes}
\usepackage{biblatex}
\addbibresource{bibliography.bib}

% Ændring af titler
\usepackage{sectsty}
\subsectionfont{\normalfont\bfseries}
\subsubsectionfont{\itshape}

%Rstudio pakker
\usepackage[svgnames]{xcolor}
\usepackage{listings}

\lstset{language=R,
basicstyle=\scriptsize\ttfamily,
commentstyle=\ttfamily\color{gray},
numbers=left,
numberstyle=\ttfamily\color{gray}\footnotesize,
stepnumber=1,
numbersep=5pt,
backgroundcolor=\color{white},
showspaces=false,
showstringspaces=false,
showtabs=false,
frame=single,
tabsize=2,
captionpos=b,
breaklines=true,
breakatwhitespace=false,
title=\lstname,
escapeinside={},
keywordstyle={},
morekeywords={}
}

% fancy table
\usepackage{booktabs}
\usepackage{multirow}
\pagestyle{fancy}
\fancyhf{}
\fancyhead[L]{\leftmark}
\fancyhead[R]{\rightmark}
\lfoot{\author}
\rfoot{\thepage}
\renewcommand{\headrulewidth}{0.4pt}
\renewcommand{\footrulewidth}{0.4pt}

%symbolforkortelser
\newcommand{\lll}{\mathcal{L}}
\newcommand{\LL}{\Leftrightarrow}
\newcommand{\lp}{\left(}
\newcommand{\rp}{\right)}
\newcommand{\rb}{\right]}
\newcommand{\lb}{\left[}
\newcommand{\lc}{\left\{}
\newcommand{\rc}{\right\}}
\newcommand{\dd}{\mathrm{d}}
\newcommand{\cc}{\mathbb{C}}
\newcommand{\aaa}{\mathbf{A}}
\newcommand{\ee}{\mathbf{E}}
\newcommand{\ff}{\mathcal{F}}
\newcommand{\ggg}{\mathcal{G}}
\newcommand{\hh}{\mathcal{H}}
\newcommand{\pp}{\mathcal{P}}
\newcommand{\qq}{\mathcal{Q}}
\newcommand{\vv}{\mathbf{V}}
\newcommand{\rr}{\mathbf{R}}
\newcommand{\nn}{\mathbf{N}}
\newcommand{\zz}{\mathbf{Z}}
\newcommand{\yy}{\mathbf{Y}}
\newcommand{\nnn}{\mathcal{N}}
\newcommand{\ii}{\mathbf{1}}
\newcommand{\bs}{\text{BS}}
\newcommand{\sumn}{\sum_{i=1}^n}
\renewcommand{\theequation}{1.\arabic{equation}}
\DeclareMathOperator*{\argmin}{\arg\,\min}
\newcommand{\RNum}[1]{\uppercase\expandafter{\romannumeral #1\relax}}

\selectlanguage{english}

\lstset{language=R}

% Environment for theorems etc.
\newtheorem{theorem}{Theorem}[section]
\newtheorem{corollary}{Corollary}[theorem]
\newtheorem{lemma}[theorem]{Lemma}
\newtheorem{assumption}{Assumption}
\newtheorem{proof}{Proof}
\newtheorem{mydef}{Definition}

% Colour citation links (otherwise they are luminescent green)
\hypersetup{colorlinks,linkcolor={black},citecolor={black},urlcolor={black}} 

% TOC depth
\setcounter{tocdepth}{2}

\title{Note: Extending the standard Hidden Markov Model}
\author{William Gram \& Kristian Strand}
\date{\today}

\begin{document}

\pagenumbering{gobble}
\maketitle

\newpage

\pagenumbering{roman}
\rfoot{\thepage}

\tableofcontents

\newpage

\pagenumbering{arabic}
\setcounter{page}{1}

\section{Exogenous extension to HMM}
\noindent In the following section we will extend the hidden markov model (HMM) to include an exogenous regressor. The model we consider is thus on the form:
\begin{align}
    Y_t=\mu_{s_t}+\beta X_t+\epsilon_t, \quad \epsilon_t \sim N(0,\Sigma{s_t}), 
\end{align}
where $Y_t=(y_{1t},y_{2t},..,y_{At})'$ is a vector of returns at time t on the A different asset classes. $X_t$ is an exogenous regressors and $S_t$ is the state variable which, for simplicity, can be in one of 2 states at time t. Following the note "Finite State SV: EM-algorithm-based Estimation" Rahbek \& Pedersen (2017), we start out by treating the state variable as observed. We treat the exogenous variable as independent of the state such that $f(x\vert s)=f(x)$ and we assume $X_t$ to follow an AR(1) process. In the following, $\theta$ is the set of parameters to be estimated. The likelihood function can then be computed as (where we ignore the initial density at first):
\begin{align}
    &f_\theta(Y_{1:T},S_{1:T},X_{1:T})=f_\theta(y_1,...,y_T,s_1,...,s_T,X_1...,X_T) \notag \\
    =&f_\theta(Y_T\vert S_{1:T},X_{1:T},Y_{1:T-1})f_\theta(S_{1:T},X_{1:T},Y_{1:T-1}) \notag \\
    =&f_\theta(Y_T\vert S_T,X_T)f_\theta(X_T\vert S_{1:T},X_{1:T-1},Y_{1:T-1})f_\theta(S_{1:T},X_{1:T-1},Y_{1:T-1}) \notag \\
    =&f_\theta(Y_T\vert S_T,X_T)f_\theta(X_T\vert X_{T-1})f_\theta(S_{1:T},X_{1:T-1},Y_{1:T-1}) \notag \\
    =&f_\theta(Y_T\vert S_T,X_T)f_\theta(X_T\vert X_{T-1})f_\theta(S_{T}\vert S_{1:T-1})f_\theta(S_{T-1},X_{1:T-1},Y_{1:T-1}) \notag \\\
    &\vdots \notag \\
    =&\prod_{t=2}^{T} f_\theta(Y_t\vert S_t,X_t)f_\theta(X_t\vert X_{t-1})f_\theta(S_t\vert S_{t-1}).
\end{align}
With the initial density $f(x_1,s_1,y_1)$, which can be factorized as above, we have:
\begin{align}\label{likelihoodfct}
    =f(y_1\vert s_1,x_1)f(x_1)f(s_1)\prod_{t=2}^{T} f_\theta(Y_t\vert S_t,X_t)f_\theta(X_t\vert X_{t-1})f_\theta(S_t\vert S_{t-1}).
\end{align}
We now need to evaluate the densities in (1.3):
\begin{enumerate}
    \item $f_\theta(Y_t\vert S_t,X_t)=f(y_t\vert x_t,s_t=1)^{\mathbb{1}_{s_t=1}}f(y_t\vert x_t,s_t=2)^{\mathbb{1}_{s_t=1}}$
    \item $f_\theta(X_t\vert X_{t-1})=\frac{1}{\sqrt{2\pi\sigma_x^2}}\exp{\{-\frac{1}{2}(\frac{x_t-\rho x_{t-1}}{\sigma_x})^2\}}$, assuming $x_t$ follows and AR(1) process
    \item $f_\theta(S_t\vert S_{t-1})=p_{11}^{\mathbb{1}_{11}}p_{22}^{\mathbb{1}_{22}}p_{12}^{\mathbb{1}_{12}}p_{21}^{\mathbb{1}_{21}}$
\end{enumerate}
At time t, (\ref{likelihoodfct}) can then be computed as (ignoring the initial density): 
\begin{align}\label{likelihoodcon}
    &f(Y_t\vert S_t,X_t)f(X_t\vert X_{t-1})f(S_t\vert S_{t-1}) = f(y_t\vert x_t,s_t=1)^{\mathbb{1}_{s_t=1}}f(y_t\vert x_t,s_t=2)^{\mathbb{1}_{s_t=2}}f(x_t\vert x_{t-1})p_{11}^{\mathbb{1}_{11}}p_{22}^{\mathbb{1}_{22}}p_{12}^{\mathbb{1}_{12}}p_{21}^{\mathbb{1}_{21}} \notag \\ 
    &= f(x_t\vert x_{t-1}) \lb p_{11}f(y_t\vert x_t,s_t=1) \rb^{\mathbb{1}_{11}} \lb p_{22}f(y_t\vert x_t,s_t=2) \rb^{\mathbb{1}_{22}} \lb p_{12}f(y_t\vert x_t,s_t=2) \rb^{\mathbb{1}_{12}} \lb p_{21}f(y_t\vert x_t,s_t=1) \rb^{\mathbb{1}_{21}}  \notag \\
    &=f(x_t\vert x_{t-1})\prod_{i=1}^{2}\lb p_{i1}f(y_t\vert x_t,s_t=1) \rb^{\mathbb{1}_{i2}}\lb p_{1i}f(y_t\vert x_t,s_t=1) \rb^{\mathbb{1}_{i2}}  \notag \\
    &=f(x_t\vert x_{t-1})\prod_{i,j=1}^{2}\lb p_{ij}f(y_t\vert x_t,s_t=j) \rb^{\mathbb{1}_{ij}} 
\end{align}
The likelihood function for all t is then found by multiplying the likelihood contribution a time t in (\ref{likelihoodcon}) for all t: 
\begin{align}\label{likelihood2stage}
    f_\theta(Y_T,S_T,X_T)=\prod_{t=2}^{T}f(x_t\vert x_{t-1})\prod_{i,j=1}^{2}\lb p_{ij}f(y_t\vert x_t,s_t=j) \rb^{\mathbb{1}_{ij}}
\end{align}
(\ref{likelihood2stage}) holds only for the 2-state model, but can easily be generalised to an N-state model:
\begin{align}\label{likelihoodNstage}
    f_\theta(Y_T,S_T,X_T;\theta)=\underbrace{\prod_{j=1}^{N}\lb f(s_1)f_\theta(y_1\vert x_1,s_1=j) \rb^{\mathbb{1}_{s_1=j}}f(x_1\vert x_0)}_{\text{Initial distribution}}\prod_{t=2}^{T}f(x_t\vert x_{t-1})\prod_{i,j=1}^{2}\lb p_{ij}f_\theta(y_t\vert x_t,s_t=j) \rb^{\mathbb{1}_{ij}}.
\end{align}
We now proceed by taking logs to (\ref{likelihoodNstage}) to obtain the full log-likelihood function in terms of $Y_T$,$S_T$ and $X_T$. This can be written as:
\begin{align}\label{loglikNstage}
    &L(Y_T,S_T,X_T;\theta)=\log(f_\theta(Y_T,S_T,X_T;\theta)) \notag \\
    &=\sum_{j=1}^{N}( \mathbb{1}_{s_1=j}(\log(f(s_1))+\log(f_\theta(y_1\vert x_1,s_1=j)))+\log(f_\theta(x_1))) \notag \\ 
    &+\sum_{t=2}^{T}(\log(f(x_t\vert x_{t-1})))+\sum_{i,j=2}^{N}(\mathbb{1}_{ij}(\log(p_{ij})+\log(f_\theta(y_t\vert x_t,s_t=j)))) \notag \\
    &= \sum_{j=1}^{N}\mathbb{1}_{s_1=j}\log(f(s_1))+\sum_{t=1}^{T}\log f_\theta(x_t\vert x_{t-1})+\sum_{t=2}^{T}\sum_{i,j=1}^{N}\log(p_{ij})\mathbb{1}_{\{s_{t-1}=i,s_t=j\}} \notag \\
    &+\sum_{t=1}^{T}\sum_{j=1}^{N}\mathbb{1}_{s_t=j}\log(f_\theta(y_t\vert x_t,s_t=j)) \notag \\
    &=c+\sum_{t=1}^{T}(\log f_\theta(x_t\vert x_{t-1})+\sum_{j=1}^{N}\mathbb{1}_j\log(f_\theta(y_t\vert x_t,s_t=j)))+\sum_{t=2}^{T}\sum_{i,j=1}^{N}\log(p_{ij})\mathbb{1}_{ij},
\end{align}
where $c$ is some positive constant as $f(s_1)$ and $f(x_1)$ are considered known.

\subsection{The E-step}
\noindent As we shall estimate parameters through the EM-algorithm, let $\Tilde{\theta}$ be an initial guess of the parameters maximising the likelihood function, and take the expectation to (\ref{loglikNstage}): \begin{align}\label{Lem}
   &L_{EM}(Y_T,X_T;\theta) = \mathbb{E}_{\Tilde{\theta}} \lb L(Y_T,S_T,X_T;\theta \vert Y_T,X_T) \rb \notag \\
   &=c+\sum_{t=1}^{T}(\log f_\theta(x_t\vert x_{t-1})+\sum_{j=1}^{N}\mathbb{E}\lb \mathbb{1}_j \rb\log(f_\theta(y_t\vert x_t,s_t=j)))+\sum_{t=2}^{T}\sum_{i,j=1}^{N}\log(p_{ij})\mathbb{E}\lb\mathbb{1}_{ij}\rb \notag \\
   &= c+\sum_{t=1}^{T}\lb \log f_\theta(x_t\vert x_{t-1})+\sum_{j=1}^{N}p_t^*(j)  \log(f_\theta(y_t\vert x_t,s_t=j))\rb+\sum_{t=2}^{T}p_t^*(i,j)\sum_{i,j=1}^{N}\log(p_{ij}),
\end{align}
where $p_t^*(i,j)=P_{\Tilde{\theta}}\lb s_{t-1}=i,s_t=j\vert Y_T,X_T \rb$ and $p_t^*(j)=P_{\Tilde{\theta}}\lb s_t=j\vert Y_T,X_T \rb$. $p_t^*(i,j)$ is is the smoothed transition probabilities while $p_t^*(j)$ is the smoothed probabilities.
 
\subsection{The M-step}

\noindent With $Y_t$ being a conditional multivariate normal distribution and $X_t$ an AR(1) process which is conditionally normal, we have:
\begin{align}
    f_\theta(Y_t\vert X_t,S_t=j)&= \frac{1}{\sqrt{2\pi^A \det(\Sigma_{S_t})}}\exp{\lc-\frac{1}{2}(Y_t-\mu_{S_t}-\beta X_t)'\Sigma_{S_t}^{-1}((Y_t-\mu_{S_t}-\beta X_t))\rc}. \\
    f_\theta(X_t\vert X_{t-1})&= \frac{1}{\sqrt{2\pi\sigma_x^2}}\exp{\lc-\frac{1}{2}\lp\frac{X_t-\rho X_{t-1}}{\sigma_x^2}\rp^2\rc},
\end{align}
where $X_t=\rho X_{t-1}+\epsilon_t^x, \quad \epsilon_t^x\sim iidN(0,\sigma_x^2)$. We now proceed by taking logs to the two expressions above: 
\begin{align}
    \log(f_\theta(Y_t\vert X_t,S_t=j))&=-\frac{1}{2}\lp A\log(2\pi)+ \log(\det(\Sigma_{S_t}))+(Y_t-\mu_{S_t}-\beta X_t )'\Sigma_{S_t}^{-1}((Y_t-\mu_{S_t}-\beta X_t)) \rp \\
    \log(f_\theta(X_t\vert X_{t-1}))&=-\frac{1}{2}\lp \log( (2\pi\sigma_x^2))+\lp\frac{X_t-\rho X_{t-1}}{\sigma_x^2}\rp^2 \rp.
\end{align}
Plugging the two expressions into (\ref{Lem}) we can now maximise (\ref{Lem}) over $\theta$ keeping $p_t^*(i,j)$ and $p_t^*(j)$ constant and hence find the parameter estimates.    

\subsection{Computing $p_t^*(i,j)$ and $p_t^*(j)$ - The forward-backward algorithm}


\noindent In the following we will modify the forward-backward algorithm to take the exogenous variable into account, when computing $p_t^*(i,j)$ and $p_t^*(j)$. Observe that for any t,
\begin{align}
    &L(Y_T,X_t;\theta)=\sum_{i=1}^{N}f(Y_{t+1:T},X_{t+1:T}\vert Y_{1:t},X_{1:t},s_t=i;\theta)f(Y_{1:t},X_{1:t},s_t=i) \notag \\
    &=\sum_{j=1}^{N}\underbrace{f(Y_{t+1:T},X_{t+1:T}\vert X_t,s_t=i;\theta)}_{b_t(i)}\underbrace{f(Y_{1:t},X_{1:t},s_t=i)}_{a_t(i)}.
\end{align}
The main idea here is, that we cab use $a_t$ and $b_t$ to compute the smoothed probabilities, which we shall see soon. Before that we compute $a_t(j)$ and $b_t(j)$ on a recursive form:
\begin{align}
    a_t(i)&=f(Y_{1:t},X_{1:t},s_t=i) \notag \\
          &=\sum_{i=1}^{N}f(Y_{1:t},X_{1:t},s_{t-1}=j,s_t=i) \notag \\
          &=\sum_{i=1}^{N}f(Y_t \vert Y_{1:t-1},X_{1:t},s_{t-1}=j,s_t=i)f( Y_{1:t-1},X_{1:t},s_{t-1}=j,s_t=i) \notag \\
          &=\sum_{i=1}^{N}f(Y_t \vert X_t,s_t=j)f(X_t \vert Y_{1:t-1},X_{1:t-1},s_{t-1}=j,s_t=i)f(Y_{1:t-1},X_{1:t-1},s_{t-1}=j,s_t=i) \notag \\
          &=\sum_{i=1}^{N}f(Y_t \vert X_t,s_t=i)f(X_t \vert X_{t-1})f(Y_{1:t-1},X_{1:t-1},s_{t-1}=j,s_t=i) \notag \\
          &=\sum_{i=1}^{N}f(Y_t \vert X_t,s_t=i)f(X_t \vert X_{t-1})f(s_t=i\vert s_{t-1}=j,Y_{1:t-1},X_{1:t-1})\underbrace{f(s_{t-1}=j,Y_{1:t-1},X_{1:t-1})}_{a_{t-1}(j)} \notag \\
          &=\sum_{i=1}^{N}f(Y_t \vert X_t,s_t=i)f(X_t \vert X_{t-1})f(s_t=i\vert s_{t-1}=j)a_{t-1}(i) \notag \\
          &=\sum_{i=1}^{N}f(Y_t \vert X_t,s_t=i)f(X_t \vert X_{t-1})p_{ij}a_{t-1}(i)
\end{align}
\begin{align}
    b_t(i)&=f(Y_{t+1:T},X_{t+1:T}\vert X_t,s_t=i) \notag \\
          &=\sum_{i=1}^{N}f(S_{t+1}=i,Y_{t+1:T},X_{t+1:T}\vert X_t,s_t=j) \notag \\
          &=\sum_{i=1}^{N}f(Y_{t+1}\vert S_{t+1}=i,S_t=j,Y_{t+2:T},X_{t+1:T})f(Y_{t+2:T},X_{t+1:T},S_{t+1}=i\vert s_t=j) \notag \\
          &=\sum_{i=1}^{N}f(Y_{t+1}\vert S_{t+1}=i,X_{t+1})f(Y_{t+2:T},X_{t+2:T}\vert S_{t+1}=i,S_t=j,X_{t+1})f(X_{t+1},S_{t+1}=i\vert S_t=j) \notag \\
          &=\sum_{i=1}^{N}f(Y_{t+1}\vert S_{t+1}=i,X_{t+1})f(Y_{t+2:T},X_{t+2:T}\vert S_{t+1}=i,X_{t+1})f(X_{t+1})f(S_{t+1}=i\vert S_t=j) \notag \\
          &=\sum_{i=1}^{N}f(Y_{t+1}\vert S_{t+1}=i,X_{t+1})b_{t+1}(i)f(X_{t+1})p_{ij}
\end{align}
What is left to be shown now, is how to compute the smoothed probabilities, $p_t^*(i,j)$ and $p_t^*(j)$, using $a_t(j)$ and $b_t(j)$. Using the definition of $p_t^*(j)$ (and the law of total probability we have):
\begin{align}
    p_t^*(i)&= f(s_t=i \vert Y_{1:T},X_{1:T}) \notag \\
            &=\frac{f(s_t=i,Y_{1:T},X_{1:T})}{\sum_{j=1}^{N}f(Y_{1:T},X_{1:T},S_t=j)} \notag \\
            &=\frac{f(Y_{t+1:T},X_{t+1:T} \vert Y_{1:t},X_{1:t},S_t=i  )f(X_{1:t},Y_{1:t},S_t=i)}{\sum_{j=1}^{N}f(Y_{t+1:T},X_{t+1:T} \vert Y_{1:t},X_{1:t},S_t=j  )f(X_{1:t},Y_{1:t},S_t=j)} \notag \\
            &=\frac{f(Y_{t+1:T},X_{t+1:T} \vert X_{t},S_t=i  )f(X_{1:t},Y_{1:t},S_t=i)}{\sum_{j=1}^{N}f(Y_{t+1:T},X_{t+1:T} \vert X_{t},S_t=j  )f(X_{1:t},Y_{1:t},S_t=j)} \notag \\
            &=\frac{b_t(i)a_t(i)}{\sum_{j=1}^{N}b_t(j)a_t(j)}.
\end{align}
Finally the smoothed transition probabilities can be calculated as:
\begin{align}
    p_t^*(i,j)&=f(s_{t-1},s_t=j\vert Y_{1:T},X_{1:T}) \notag \\
              &=\frac{f(Y_{t+1:T},X_{t+1:T}\vert S_{t}=j,S_{t-1}=i,Y_{1:t},X_{1:t})f(Y_{1:t},X_{1:t},S_{t-1}=i,S_t=j)}{\sum_{k=1}^{N}f(Y_{1:T},X_{1:T},S_t=k)} \notag \\
              &=\frac{f(Y_{t+1:T},X_{t+1:T}\vert S_{t}=j,S_{t-1}=i,Y_{1:t},X_{1:t})f(Y_{1:t},X_{1:t},S_{t-1}=i,S_t=j)}{\sum_{k=1}^{N}b_t(k)a_t(k)} \notag \\
              &=\frac{f(Y_{t+1:T},X_{t+1:T}\vert S_{t}=j,X_t)f(Y_t\vert S_{t-1}=i,S_t=j,Y_{1:t-1},X_{1:t})f(S_{t-1}=i,S_t=j,Y_{1:t-1},X_{1:t})}{\sum_{k=1}^{N}b_t(k)a_t(k)} \notag \\
              &=\frac{b_t(j)f(Y_t\vert S_t=j,X_t)f(S_t=j\vert S_{t-1}=i,Y_{1:t-1},X_{1:t})f(S_{t-1}=i,Y_{1:t-1},X_{1:t})}{\sum_{k=1}^{N}b_t(k)a_t(k)} \notag \\
              &=\frac{b_t(j)f(Y_t\vert S_t=j,X_t)p_{ij}f(X_t\vert S_{t-1}=i,Y_{1:t-1},X_{1:t-1})f(S_{t-1}=i,Y_{1:t-1},X_{1:t-1})}{\sum_{k=1}^{N}b_t(k)a_t(k)} \notag \\
              &=\frac{b_t(j)f(Y_t\vert S_t=j,X_t)p_{ij}f(X_t\vert X_{t-1})a_{t-1}(i)}{\sum_{k=1}^{N}b_t(k)a_t(k)}
\end{align}
\newpage
\noindent This concludes the derivation of the exogenous extension to the HMM model. Importantly the forward-backward algorithm as well as the likelihood function has been modified to allow for the exogenous regressor.  

\end{document}