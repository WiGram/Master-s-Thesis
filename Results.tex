% !TeX spellcheck = da_DK
\documentclass[11pt,a4paper,oneside]{article}

% Overordnet opsætning
\setlength\parindent{24pt}
\setlength\parskip{3pt}
\setlength{\headheight}{14pt}
\renewcommand{\baselinestretch}{1.5}

\usepackage[utf8]{inputenc}
\usepackage[danish,english]{babel}
\usepackage{amsfonts,amsmath} %math: align og gather mm.; fonts: flere forskellige symboler
\usepackage[left=2cm, right=2cm, bottom=2cm]{geometry}
\usepackage{enumitem} %anvendes til ændring i itemize mm.
\usepackage{fancyhdr} %Opsaening af sidehoved og -fod
\usepackage{lastpage} % Side "x af X".
\usepackage[ocgcolorlinks,linkcolor=black]{hyperref}
\usepackage{indentfirst} %"Indent" efter section og chapters etc.
\usepackage{graphicx} % for plotting graphs in matrix
\usepackage{subcaption} % figure "footnotes"
\usepackage[font=small,compatibility=false]{caption}
\usepackage[nottoc,numbib]{tocbibind}    

% Ændring af titler
\usepackage{sectsty}
\subsectionfont{\normalfont\bfseries}
\subsubsectionfont{\itshape}

%Rstudio pakker
\usepackage[svgnames]{xcolor}
\usepackage{listings}

\lstset{language=R,
basicstyle=\scriptsize\ttfamily,
commentstyle=\ttfamily\color{gray},
numbers=left,
numberstyle=\ttfamily\color{gray}\footnotesize,
stepnumber=1,
numbersep=5pt,
backgroundcolor=\color{white},
showspaces=false,
showstringspaces=false,
showtabs=false,
frame=single,
tabsize=2,
captionpos=b,
breaklines=true,
breakatwhitespace=false,
title=\lstname,
escapeinside={},
keywordstyle={},
morekeywords={}
}

% fancy table
\usepackage{booktabs}
\usepackage{multirow}

\pagestyle{fancy}
\fancyhf{}
\fancyhead[L]{\leftmark}
\fancyhead[R]{\rightmark}
\lfoot{\author}
\rfoot{\thepage}
\renewcommand{\headrulewidth}{0.4pt}
\renewcommand{\footrulewidth}{0.4pt}

%symbolforkortelser
\newcommand{\lll}{\mathcal{L}}
\newcommand{\LL}{\Leftrightarrow}
\newcommand{\lp}{\left(}
\newcommand{\rp}{\right)}
\newcommand{\rb}{\right]}
\newcommand{\lb}{\left[}
\newcommand{\lc}{\left\{}
\newcommand{\rc}{\right\}}
\newcommand{\dd}{\mathrm{d}}
\newcommand{\cc}{\mathbb{C}}
\newcommand{\ee}{\mathbf{E}}
\newcommand{\ff}{\mathcal{F}}
\newcommand{\ggg}{\mathcal{G}}
\newcommand{\hh}{\mathcal{H}}
\newcommand{\pp}{\mathcal{P}}
\newcommand{\qq}{\mathcal{Q}}
\newcommand{\vv}{\mathbf{V}}
\newcommand{\rr}{\mathbf{R}}
\newcommand{\nn}{\mathbf{N}}
\newcommand{\nnn}{\mathcal{N}}
\newcommand{\ii}{\mathbf{1}}
\newcommand{\bs}{\text{BS}}
\newcommand{\sumn}{\sum_{i=1}^n}
\renewcommand{\theequation}{1.\arabic{equation}}
\DeclareMathOperator*{\argmin}{\arg\,\min}
\newcommand{\RNum}[1]{\uppercase\expandafter{\romannumeral #1\relax}}

\selectlanguage{english}

\lstset{language=R}

% Environment for theorems etc.
\newtheorem{theorem}{Theorem}[section]
\newtheorem{corollary}{Corollary}[theorem]
\newtheorem{lemma}[theorem]{Lemma}
\newtheorem{assumption}{Assumption}
\newtheorem{proof}{Proof}
\newtheorem{mydef}{Definition}

% Colour citation links (otherwise they are luminescent green)
\hypersetup{colorlinks,linkcolor={black},citecolor={black},urlcolor={black}} 


\title{Monte Carlo Methods in Financial Engineering}
\author{William Gram \& Kristian Strand}
\date{November 2018}

\begin{document}







\section{Results}
\subsection*{Results for European options}


\begin{table}[ht]
\centering
\captionsetup{justification=centering,margin=0.6cm}
\caption{\textbf{European call option prices with K=110}}
\resizebox{\textwidth}{!}{
\begin{tabular}{c|cccccc}
\toprule
    \textbf{N}                   &   
    \textbf{CMC}                 &  
    \textbf{\begin{tabular}[c]{@{}c@{}} Antithetic \\  Variates \end{tabular}}               &  
    \textbf{\begin{tabular}[c]{@{}c@{}} Control \\  Variates \end{tabular}}               & 
    \textbf{\begin{tabular}[c]{@{}c@{}} Importance \\  Sampling \end{tabular}}               &   
    \textbf{\begin{tabular}[c]{@{}c@{}} Stratified \\  Sampling \end{tabular}}               & 
   \textbf{\begin{tabular}[c]{@{}c@{}} Importance + \\  Stratified \end{tabular}}                
    \\ \midrule
        1,000   &
        \begin{tabular}[c]{@{}c@{}} 6.38117    \\  (0.3805566)    \\  {[}NA{]}      \end{tabular}    &
        \begin{tabular}[c]{@{}c@{}} 5.953382    \\  (0.2192796)    \\  {[}1.74{]}    \end{tabular}    &
        \begin{tabular}[c]{@{}c@{}} 5.916865    \\  (0.1974965)    \\  {[}1.93{]}    \end{tabular}    &
        \begin{tabular}[c]{@{}c@{}} 6.270465    \\  (0.1341173)    \\  {[}2.84{]}    \end{tabular}    &
        \begin{tabular}[c]{@{}c@{}} 6.084149    \\  (0.062184)    \\  {[}6.12{]}    \end{tabular}    &
        \begin{tabular}[c]{@{}c@{}} 6.040885    \\  (0.002972671)    \\  {[}128.02{]}   \end{tabular}  
    \\ \hline
        10,000  &
        \begin{tabular}[c]{@{}c@{}} 6.007266    \\  (0.1153889)    \\  {[}NA{]}     \end{tabular}        &
        \begin{tabular}[c]{@{}c@{}} 5.918181   \\  (0.06963658)    \\  {[}1.66{]}      \end{tabular}    &
        \begin{tabular}[c]{@{}c@{}} 6.016185    \\  (0.06100779)    \\  {[}1.89{]}      \end{tabular}   &
        \begin{tabular}[c]{@{}c@{}} 6.052749    \\  (0.04297205)    \\  {[}2.69{]}      \end{tabular}   &
        \begin{tabular}[c]{@{}c@{}} 6.043807     \\  (0.003871488)    \\  {[}29.80{]}      \end{tabular} &
        \begin{tabular}[c]{@{}c@{}} 6.040014    \\  (0.0002037212)    \\  {[}566.41{]}      \end{tabular}  
    \\ \hline
        100,000 &
        \begin{tabular}[c]{@{}c@{}} 6.037891    \\  (0.03680937)    \\  {[}NA{]}    \end{tabular}   &
        \begin{tabular}[c]{@{}c@{}} 6.007418    \\  (0.02223155)    \\  {[}1.66{]}      \end{tabular}   &
        \begin{tabular}[c]{@{}c@{}} 6.029637    \\  (0.01940272)    \\  {[}1.90{]}      \end{tabular}   &
        \begin{tabular}[c]{@{}c@{}} 6.055545    \\  (0.01360269)    \\  {[}2.71{]}      \end{tabular}   &
        \begin{tabular}[c]{@{}c@{}} 6.040186    \\  (0.0003034668)    \\  {[}121.30{]}      \end{tabular} &
        \begin{tabular}[c]{@{}c@{}} 6.040075    \\  (0.0000172579)    \\  {[}2132.9{]}      \end{tabular}  
    \\ \hline
    1,000,000  &
        \begin{tabular}[c]{@{}c@{}} 6.058021    \\  (0.01166503)    \\  {[}NA{]}    \end{tabular}   &
        \begin{tabular}[c]{@{}c@{}} 6.06606   \\  (0.00705725)    \\  {[}1.65{]}      \end{tabular}   &
        \begin{tabular}[c]{@{}c@{}} 6.052448    \\  (0.006132357)    \\  {[}1.90{]}      \end{tabular}   &
        \begin{tabular}[c]{@{}c@{}} 6.036932    \\  (0.004305661)    \\  {[}2.71{]}      \end{tabular}   &
        \begin{tabular}[c]{@{}c@{}} 6.040054    \\  (0.00003736036)    \\  {[}312.23{]}      \end{tabular} &
        \begin{tabular}[c]{@{}c@{}} 6.040088   \\  (0.0000008076588)    \\  {[}14443.02{]}      \end{tabular}  
         \\ \hline
\end{tabular}}
\caption*{\small{\textit{Standard deviation in parenthesis and factor reduction relative to CMC in brackets (defined as $\frac{SD_{CMC}}   {SD_{method}}$). Parameters: $S_0$ = 100, \textbf{K=110}, $r = 5\%$, $\sigma = 20\%$, T=1, \textbf{Analytic Black-Scholes price = 6.040088}}}}
\vspace{-5mm}
\end{table}

\noindent The results for European options with strike K=110 are presented in the table above. The analytic black-scholes price is 6.040088 which we can use to get and indication of whether our simulation results are correct. All methods does converge to the analytic price while the standard deviation is reduced when the amount of simulation is increased.     

In terms of variance reduction, it is pretty clear that the method resulting in largest variance reduction is the combination of importance sampling and stratified sampling. With only 1000 simulations the standard deviation is reduced by a factor of 128 and with 1,000,000 simulation that factor is 14,443! Looking at each method individually stratified sampling is most effective while the three other methods are only reducing the standard deviation with factors around 1.5-3.0. However, in terms of figuring out which method is actually most effective one must also consider the computational aspect of each method. For example when the amount of simulations is increase, the amount of strata is also increased in the way we implemented stratified sampling. It would be an unfair comparison if computational time is disregarded. 

In table two below the computational time and efficiency is presented. Time is denoted in seconds and the computational efficiency is defined as the standard deviation multiplied with the computational time in seconds and can be seen in parenthesis in table XXX. Additionally, table XXX presents, in brackets, the computational efficiency of the given method divided by the computational efficiency of the crude Monte Carlo. Hence we are looking for the method which has the smallest ratio as this is the most efficient in terms of standard deviations reduction taking computational time into account.

Importance sampling combined with stratified sampling is still the superior method for any amount of simulation. However, results are mixed when comparing the methods individually. For example for only 1,000 simulations the control variate method is to be preferred (with a ratio of 0.5189), while for 10,000 simulations importance sampling is preferred (with a ratio of 0.3724). For 100,000 simulations and more the stratified sampling is preferred. Interestingly for 1,000,000 simulations the next best method is actually the simple Antithetic varaites (when excluding importance + stratified). In appendix XXX we show the same two tables but for K=150 i.e. considering deep out-of-the money options. The conclusions are unchanged?!?!?!  


%%From earlier comments to the result section: 

%Having introduced the variance reduction methods, we now consider the results. Stratified sampling is the most effective of our implementations in reducing the variance of the estimator for the European call option with $S=100$ and $K=110$. With only 1,000 simulations, the call option price is estimated to be 6.039 with a variance of 0.008 which is very close to the analytic price of 6.04. To check the robustness of our results, we run the simulations for deep out the money options (DOTM), where $K=150$ and the remaining parameters are left unchanged. Results are presented in table 2 below.



\begin{table}[ht]
\centering
\captionsetup{justification=centering,margin=0.6cm}
\caption{\textbf{Computational time and efficiency}}
\resizebox{\textwidth}{!}{
\begin{tabular}{c|cccccc}
\toprule
   \textbf{N}                   &   
    \textbf{CMC}                 &  
    \textbf{\begin{tabular}[c]{@{}c@{}} Antithetic \\  Variates \end{tabular}}               &  
    \textbf{\begin{tabular}[c]{@{}c@{}} Control \\  Variates \end{tabular}}               & 
    \textbf{\begin{tabular}[c]{@{}c@{}} Importance \\  Sampling \end{tabular}}               &   
    \textbf{\begin{tabular}[c]{@{}c@{}} Stratified \\  Sampling \end{tabular}}               & 
   \textbf{\begin{tabular}[c]{@{}c@{}} Importance + \\  Stratified \end{tabular}} 
    \\ \midrule
        1,000   &
        \begin{tabular}[c]{@{}c@{}} 0.0009970665    \\  (0.0003794402) \\ {[}n.m.{]} \end{tabular}   &
        \begin{tabular}[c]{@{}c@{}} 0.005983114   \\  (0.001311975)   \\ {[}3.45766{]}  \end{tabular}   &
        \begin{tabular}[c]{@{}c@{}} 0.0009970665    \\  (0.0001969171) \\ {[}0.5189674{]}  \end{tabular}   &
        \begin{tabular}[c]{@{}c@{}} 0.008975983    \\  (0.001203835) \\    {[}3.17266{]}   \end{tabular}   &
        \begin{tabular}[c]{@{}c@{}} 0.01097107    \\  (0.000682225) \\    {[}1.797978{]}    \end{tabular}   &
        \begin{tabular}[c]{@{}c@{}} 0.002991915    \\  (0.000008893978) \\ {[}0.02343974{]}  
        \end{tabular}
    \\ \hline
        10,000  &
        \begin{tabular}[c]{@{}c@{}} 0.001995087    \\  (0.0002302109) \\ {[}n.m.{]}       \end{tabular}   &
        \begin{tabular}[c]{@{}c@{}} 0.006980896    \\  (0.0004861257) \\ {[}2.111654{]}   \end{tabular}   &
        \begin{tabular}[c]{@{}c@{}} 0.005984068    \\  (0.0003650748)  \\ {[}1.585828{]} \end{tabular}   &
        \begin{tabular}[c]{@{}c@{}} 0.001995087    \\  (0.00008573296)   \\ {[}0.3724106{]} \end{tabular}   &
        \begin{tabular}[c]{@{}c@{}} 0.03490686     \\  (0.0001351415) \\ {[}0.5870335{]} \end{tabular}  &
        \begin{tabular}[c]{@{}c@{}} 0.03490615    \\  (0.000007111121) \\ {[}0.0308896{]}   \end{tabular}
    \\ \hline
        100,000 &
        \begin{tabular}[c]{@{}c@{}} 0.01296592    \\  (0.0004772673)  \\ {[}n.m.{]}  \end{tabular}   &
        \begin{tabular}[c]{@{}c@{}} 0.01096988    \\  (0.0002438773)  \\ {[}0.510987{]} \end{tabular}   &
        \begin{tabular}[c]{@{}c@{}} 0.01396322    \\  (0.0002709245)  \\ {[}0.5676579{]} \end{tabular}   &
        \begin{tabular}[c]{@{}c@{}} 0.0169549    \\  (0.0002306322) \\ {[}0.4832349{]}   \end{tabular}   &
        \begin{tabular}[c]{@{}c@{}} 0.353055   \\  (0.0001071405)   \\ {[}0.2244874{]} \end{tabular}   &
        \begin{tabular}[c]{@{}c@{}} 0.349067    \\  (0.000006024163) \\ {[}0.0126222{]}   \end{tabular}
    \\ \hline
    1,000,000  &
        \begin{tabular}[c]{@{}c@{}} 0.131649    \\  (0.00153569)   \\ {[}n.m.{]}    \end{tabular}   &
        \begin{tabular}[c]{@{}c@{}} 0.1097059    \\  (0.0007742221)   \\ {[}0.5041525{]}   \end{tabular}   &
        \begin{tabular}[c]{@{}c@{}} 0.1645601    \\  (0.001009141)   \\ {[}0.6571254{]}   \end{tabular}   &
        \begin{tabular}[c]{@{}c@{}} 0.210438    \\  (0.0009060747)    \\ {[}0.5900113{]}   \end{tabular}   &
        \begin{tabular}[c]{@{}c@{}} 3.442796    \\  (0.0001286241) \\ {[}0.08375653{]} \end{tabular}   &
        \begin{tabular}[c]{@{}c@{}} 3.471719   \\  (0.000002803965)   \\ {[}0.001825866{]} \end{tabular}
         \\ \hline
\end{tabular}}
\caption*{\small{\textit{Time denoted in seconds. Computational efficiency, defined as standard error multiplied with time in seconds, in parenthesis and computational efficiency relative to CMC in brackets. Parameters: $S_0$ = 100, \textbf{K=110}, $r = 5\%$, $\sigma = 20\%$, T=1}}}
\vspace{-10mm}
\end{table}



\clearpage


\subsection*{Results for arithemtic Asian options}

\begin{table}[ht]
\centering
\captionsetup{justification=centering,margin=0.6cm}
\caption{\textbf{Arithmetic Asian call option prices}}
\resizebox{\textwidth}{!}{
\begin{tabular}{c|cccccc}
\toprule
    \textbf{$\sigma$}                   &   
    \textbf{CMC}                 &  
    \textbf{\begin{tabular}[c]{@{}c@{}} Antithetic \\  Variates \end{tabular}}               &  
    \textbf{\begin{tabular}[c]{@{}c@{}} Control \\  Variates \end{tabular}}               & 
    \textbf{\begin{tabular}[c]{@{}c@{}} Importance \\  Sampling \end{tabular}}               &   
    \textbf{\begin{tabular}[c]{@{}c@{}} Stratified \\  Sampling \end{tabular}}               & 
   \textbf{\begin{tabular}[c]{@{}c@{}} Importance + \\  Stratified \end{tabular}} 
    \\ \midrule
        0.20   &
        \begin{tabular}[c]{@{}c@{}} 2.036118    \\  (0.04999434)    \\  {[}n.m{]}    \end{tabular}   &
        \begin{tabular}[c]{@{}c@{}} 2.023902    \\  (0.03168414)    \\  {[}1.58{]}      \end{tabular}   &
        \begin{tabular}[c]{@{}c@{}} 1.98906    \\  (0.001954289)    \\  {[}25.58{]}      \end{tabular}   &
        \begin{tabular}[c]{@{}c@{}} 1.966753    \\  (0.01365927)    \\  {[}3.66{]}      \end{tabular}   &
        \begin{tabular}[c]{@{}c@{}} 1.98527    \\  (0.02875305)    \\  {[}1.74{]}      \end{tabular}  &
        \begin{tabular}[c]{@{}c@{}} 1.988756    \\  (0.005562187)    \\  {[}8.99{]}      \end{tabular}
    \\ \hline
        0.30  &
        \begin{tabular}[c]{@{}c@{}} 4.140628    \\  (0.09163278)    \\  {[}n.m.{]}    \end{tabular}   &
        \begin{tabular}[c]{@{}c@{}} 4.136603    \\  (0.0566799)    \\  {[}1.62{]}      \end{tabular}   &
        \begin{tabular}[c]{@{}c@{}} 4.071361    \\  (0.004540363)    \\  {[}20.18{]}      \end{tabular}   &
        \begin{tabular}[c]{@{}c@{}} 4.101059    \\  (0.03859326)    \\  {[}2.37{]}      \end{tabular}   &
        \begin{tabular}[c]{@{}c@{}} 4.066929     \\  (0.05123817)    \\  {[}1.79{]}      \end{tabular} &
        \begin{tabular}[c]{@{}c@{}} 4.061642    \\  (0.02157475)    \\  {[}4.25{]}      \end{tabular}
    \\ \hline
        0.40 &
        \begin{tabular}[c]{@{}c@{}} 6.374998    \\  (0.1372983)    \\  {[}n.m.{]}    \end{tabular}   &
        \begin{tabular}[c]{@{}c@{}} 6.372892    \\  (0.08419748)    \\  {[}1.63{]}      \end{tabular}   &
        \begin{tabular}[c]{@{}c@{}} 6.275489    \\  (0.008488484)    \\  {[}16.17{]}      \end{tabular}   &
        \begin{tabular}[c]{@{}c@{}} 6.70328   \\  (0.1932342)    \\  {[}0.71{]}      \end{tabular}   &
        \begin{tabular}[c]{@{}c@{}} 6.259439    \\  (0.07572546)    \\  {[}1.81{]}      \end{tabular} &
        \begin{tabular}[c]{@{}c@{}} 6.23073    \\  (0.06907178)    \\  {[}1.99{]}      \end{tabular}
    \\ \hline
\end{tabular}}
\caption*{\small{\textit{Standard deviation in parenthesis and factor reduction relative to CMC in brackets. Parameters: $S_0$ = 100, $K=110$, $r = 5\%$, $T=1$, $m=500$ time steps and 10,000 simulations }}}
\end{table}

\noindent Table XXX above shows the results for implementation of the various variance reduction methods for arithmetic Asian call options. Once again the standard deviation is in parenthesis and the reduction of standard deviation relative to the CMC method is in brackets.

As expected due to the obvious choice of control variate, the control variate method provides a very large reduction in standard deviation. using the geometric option price as control provides a large correlation with the arithemtic option and hence the reduction in standard deviation is as high as 25.58 times, for volatility of 20\%. Interestingly, the efficiency of importance sampling is very low and for high volatility options it does no longer provide any reduction. The stratified sampling on its own does not really depend on volatility, but the reduction factor is a merely 1.8. The combination of importance sampling and stratified sampling provides a relatively strong reduction of roughly 9 times for low volatlity options, while the effectiveness decreases significantly for higher volatility options. 

As for the European options, for a fair comparison of methods we need to consider the computational time and efficiency as well. In table XXX below the computation time in seconds in presented together with the computational effciency (again defined as the standard deviation multiplied with the computational time in seconds and can be seen in parenthesis). In brackets we show the computational efficiency of each method relatively to the computational efficiency of the CMC method. For any level of volatility it is the control variate method which is the superior method. Hence we conclude that control variates provides the most efficient way of reducing variance for arithmetic Asian call options (out of the different methods considered here).    



\begin{table}[ht]
\centering
\captionsetup{justification=centering,margin=0.6cm}
\caption{\textbf{Computational time and efficiency}}
\resizebox{\textwidth}{!}{
\begin{tabular}{c|cccccc}
\toprule
    \textbf{$\sigma$}                   &   
    \textbf{CMC}                 &  
    \textbf{\begin{tabular}[c]{@{}c@{}} Antithetic \\  Variates \end{tabular}}               &  
    \textbf{\begin{tabular}[c]{@{}c@{}} Control \\  Variates \end{tabular}}               & 
    \textbf{\begin{tabular}[c]{@{}c@{}} Importance \\  Sampling \end{tabular}}               &   
    \textbf{\begin{tabular}[c]{@{}c@{}} Stratified \\  Sampling \end{tabular}}               & 
   \textbf{\begin{tabular}[c]{@{}c@{}} Importance + \\  Stratified \end{tabular}} 
    \\ \midrule
        0.20   &
        \begin{tabular}[c]{@{}c@{}} 0.661232   \\  (0.03305785)   \\  {[}n.m.{]} \end{tabular}   &
        \begin{tabular}[c]{@{}c@{}} 0.605381    \\  (0.01918098) \\  {[}0.5802{]}     \end{tabular}   &
        \begin{tabular}[c]{@{}c@{}} 0.857707    \\  (0.001676208) \\  {[}0.0507{]}  \end{tabular}   &
        \begin{tabular}[c]{@{}c@{}} 2.529234    \\  (0.0345475)  \\  {[}1.0451{]} \end{tabular}   &
        \begin{tabular}[c]{@{}c@{}} 0.7958729    \\  (0.02288378) \\  {[}0.6922{]}   \end{tabular} &
        \begin{tabular}[c]{@{}c@{}} 4.258615    \\  (0.02368721)    \\  {[}0.7165{]}      \end{tabular}
    \\ \hline
        0.30  &
        \begin{tabular}[c]{@{}c@{}} 0.698133   \\  (0.06397187)   \\  {[}n.m.{]}    \end{tabular}   &
        \begin{tabular}[c]{@{}c@{}} 0.548039    \\  (0.03106279)  \\  {[}0.4856{]}  \end{tabular}   &
        \begin{tabular}[c]{@{}c@{}} 0.79687    \\  (0.003618079) \\  {[}0.0566{]} \end{tabular}   &
        \begin{tabular}[c]{@{}c@{}} 1.942806   \\  (0.07497922)  \\  {[}1.1721{]}  \end{tabular}   &
        \begin{tabular}[c]{@{}c@{}} 0.7330401     \\  (0.03755963) \\  {[}0.5871{]}   \end{tabular} &
        \begin{tabular}[c]{@{}c@{}} 3.750972    \\  (0.08092627)    \\  {[}1.2650{]}      \end{tabular}
    \\ \hline
        0.40 &
        \begin{tabular}[c]{@{}c@{}} 0.661232         \\  (0.09078605) \\ {[}n.m.{]}   \end{tabular}   &
        \begin{tabular}[c]{@{}c@{}} 0.5425498   \\  (0.04568133)  \\  {[}0.5032{]}    \end{tabular}   &
        \begin{tabular}[c]{@{}c@{}} 0.7918832    \\  (0.006721888) \\  {[}0.0740{]}     \end{tabular}   &
        \begin{tabular}[c]{@{}c@{}} 1.828113    \\  (0.353254) \\  {[}3.8911{]}     \end{tabular}   &
        \begin{tabular}[c]{@{}c@{}} 0.7250619    \\  (0.05490565) \\  {[}0.6048{]}     \end{tabular}
        &
        \begin{tabular}[c]{@{}c@{}} 3.69213    \\  (0.255022)    \\  {[}2.8090{]}      \end{tabular}
    \\ \hline

\end{tabular}}
\caption*{\small{\textit{Time denoted in seconds. Computational efficiency, defined as standard error multiplied with time in seconds, in parenthesis. Parameters: $S_0$ = 100, $K=110$, $r = 5\%$, $T=1$, $m=500$ time steps and 10,000 simulations }}}
\end{table}

\noindent Given the effectiveness of the control variates methods with only one control variate, we now turn to reducing variance using more than one control variate. In the table below results are presented. Control Variate \RNum{1} is the one already considered. For \RNum{2}-\RNum{4} we add control variates in addition to the geometric Asian option. Control Variates \RNum{2} includes the stock itself, Control Variates \RNum{3} includes the european call option while Control Variates \RNum{4} includes all 3 control variates, i.e. the geometric Asian call option, the stock price and the European call option.   

The table below illustrates that including more control variates, which are correlated to the arithmetic Asian option, decreases the variance and thus the standard deviation. We can also deduct from the results, that it is better to combine the geometric Asian option with the European option rather than with the stock itself. For volatility at 20\% this leads to a reduction in standard deviation of 30.76x whereas with the stock as additional control the reduction is 27.45x. However, purely in terms of standard deviation reduction, a combination of all three controls are to be preferred. This leads to reduction factors as high as 30.82x. Once again to make a fair comparison, we consider the computational efficiency. 


\begin{table}[ht]
\centering
\captionsetup{justification=centering,margin=0.6cm}
\caption{\textbf{Arithmetic Asian call option prices}}

\begin{tabular}{c|cccccc}
\toprule
    \textbf{$\sigma$}            &   
    \textbf{CMC}                 &  
    \textbf{\begin{tabular}[c]{@{}c@{}} Control \\  Variates \RNum{1} \end{tabular}}               &  
    \textbf{\begin{tabular}[c]{@{}c@{}} Control \\  Variates \RNum{2} \end{tabular}}               & 
    \textbf{\begin{tabular}[c]{@{}c@{}} Control \\  Variates \RNum{3} \end{tabular}}               & 
    \textbf{\begin{tabular}[c]{@{}c@{}} Control  \\  Variates \RNum{4} \end{tabular}}               
    \\ \midrule
        0.20   &
        \begin{tabular}[c]{@{}c@{}} 2.036118   \\  (0.04999434)   \\  {[}n.m.{]} \end{tabular}   &
        \begin{tabular}[c]{@{}c@{}} 1.98906    \\  (0.001954289) \\  {[}25.58{]}     \end{tabular}   &
        \begin{tabular}[c]{@{}c@{}} 1.989436    \\  (0.001821029) \\  {[}27.45{]}  \end{tabular}   &
        \begin{tabular}[c]{@{}c@{}} 1.988992    \\  (0.001625272)  \\  {[}30.76{]} \end{tabular} &
        \begin{tabular}[c]{@{}c@{}} 1.988911    \\  (0.001622335)  \\  {[}30.82{]} \end{tabular} 
    \\ \hline
        0.30  &
        \begin{tabular}[c]{@{}c@{}} 4.140628   \\  (0.09163278)   \\  {[}n.m.{]}    \end{tabular}   &
        \begin{tabular}[c]{@{}c@{}} 4.071361    \\  (0.004540363)  \\  {[}20.18{]}  \end{tabular}   &
        \begin{tabular}[c]{@{}c@{}} 4.071923    \\  (0.004201791) \\  {[}21.81{]} \end{tabular}   &
        \begin{tabular}[c]{@{}c@{}} 4.070449   \\  (0.00375823)  \\  {[}24.38{]}  \end{tabular}  &
        \begin{tabular}[c]{@{}c@{}} 4.070241    \\  (0.003750754)  \\  {[}24.43{]} \end{tabular} 
    \\ \hline
        0.40 &
        \begin{tabular}[c]{@{}c@{}} 6.374998       \\  (0.1372983) \\ {[}n.m.{]}   \end{tabular}   &
        \begin{tabular}[c]{@{}c@{}} 6.275489   \\  (0.008488484)  \\  {[}16.17{]}    \end{tabular}   &
        \begin{tabular}[c]{@{}c@{}} 6.276182    \\  (0.0077866418) \\  {[}17.63{]}     \end{tabular}   &
        \begin{tabular}[c]{@{}c@{}} 6.273078    \\  (0.006995914) \\  {[}19.63{]}     \end{tabular} &
        \begin{tabular}[c]{@{}c@{}} 6.272648    \\  (0.006980183)  \\  {[}19.67{]} \end{tabular} 
    \\ \hline

\end{tabular}
\caption*{\small{\textit{Standard deviation in parenthesis and factor reduction relative to CMC in brackets. Parameters: $S_0$ = 100, $K=110$, $r = 5\%$, $T=1$, $m=500$ time steps and 10,000 simulations }}}
\end{table}


\begin{table}[ht]
\centering
\captionsetup{justification=centering,margin=0.6cm}
\caption{\textbf{Computational time and efficiency}}

\begin{tabular}{c|cccccc}
\toprule
    \textbf{$\sigma$}            &   
    \textbf{CMC}                 &  
    \textbf{\begin{tabular}[c]{@{}c@{}} Control \\  Variates \RNum{1} \end{tabular}}               &  
    \textbf{\begin{tabular}[c]{@{}c@{}} Control \\  Variates \RNum{2} \end{tabular}}               & 
    \textbf{\begin{tabular}[c]{@{}c@{}} Control \\  Variates \RNum{3} \end{tabular}}               & 
    \textbf{\begin{tabular}[c]{@{}c@{}} Control  \\  Variates \RNum{4} \end{tabular}}               
    \\ \midrule
        0.20   &
        \begin{tabular}[c]{@{}c@{}} 0.6592369   \\  (0.03295811)   \\  {[}n.m.{]} \end{tabular}   &
        \begin{tabular}[c]{@{}c@{}} 0.801857    \\  (0.001567061) \\  {[}0.0475{]}     \end{tabular}   &
        \begin{tabular}[c]{@{}c@{}} 0.7988639    \\  (0.001454754) \\  {[}0.0441{]}  \end{tabular}   &
        \begin{tabular}[c]{@{}c@{}} 0.7938781    \\  (0.001290268)  \\  {[}0.0391{]} \end{tabular} &
        \begin{tabular}[c]{@{}c@{}} 0.792881    \\  (0.001286319)  \\  {[}0.0390{]} \end{tabular} 
    \\ \hline
        0.30  &
        \begin{tabular}[c]{@{}c@{}} 0.676193   \\  (0.06196145)   \\  {[}n.m.{]}    \end{tabular}   &
        \begin{tabular}[c]{@{}c@{}} 0.794874    \\  (0.003609017)  \\  {[}0.0582{]}  \end{tabular}   &
        \begin{tabular}[c]{@{}c@{}} 0.79687    \\  (0.003348282) \\  {[}0.0540{]} \end{tabular}   &
        \begin{tabular}[c]{@{}c@{}} 0.805846   \\  (0.003028555)  \\  {[}0.0489{]}  \end{tabular}  &
        \begin{tabular}[c]{@{}c@{}} 0.805846   \\  (0.00302253)  \\  {[}0.0488{]} \end{tabular} 
    \\ \hline
        0.40 &
        \begin{tabular}[c]{@{}c@{}} 0.661232  \\  (0.09078605) \\ {[}n.m.{]}   \end{tabular}   &
        \begin{tabular}[c]{@{}c@{}} 0.7918832   \\  (0.006721888) \\  {[}0.0740{]}    \end{tabular}   &
        \begin{tabular}[c]{@{}c@{}} 0.8028529   \\  (0.006251527) \\  {[}0.0689{]}     \end{tabular}   &
        \begin{tabular}[c]{@{}c@{}} 0.8138251   \\  (0.00569345) \\  {[}0.0627{]}     \end{tabular} &
        \begin{tabular}[c]{@{}c@{}} 0.7938769   \\  (0.005541406) \\  {[}0.0610{]} \end{tabular} 
    \\ \hline

\end{tabular}
\caption*{\small{\textit{Time denoted in seconds. Computational efficiency, defined as standard error multiplied with time in seconds, in parenthesis. Parameters: $S_0$ = 100, $K=110$, $r = 5\%$, $T=1$, $m=500$ time steps and 10,000 simulations }}}
\end{table}


From table XXX we conclude that the most efficient method is the combination of all three controls. For all levels of volatility the computational efficiency (standard deviation times computational time in seconds) of the combined method is smallest. Note that the method combining all three controls does not necessarily take longer to compute, and when it does take longer, the reduction in standard deviation relative to the other methods will counter the increased computational time.     





\end{document}

