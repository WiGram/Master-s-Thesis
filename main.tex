% !TeX spellcheck = da_DK
\documentclass[11pt,a4paper,oneside]{article}

% Overordnet opsætning
\setlength\parindent{24pt}
\setlength\parskip{3pt}
\setlength{\headheight}{14pt}
\renewcommand{\baselinestretch}{1.5}

\usepackage[utf8]{inputenc}
\usepackage[danish,english]{babel}
\usepackage{amsfonts,amsmath} %math: align og gather mm.; fonts: flere forskellige symboler
\usepackage[left=2cm, right=2cm, bottom=2cm]{geometry}
\usepackage{enumitem} %anvendes til ændring i itemize mm.
\usepackage{fancyhdr} %Opsaening af sidehoved og -fod
\usepackage{lastpage} % Side "x af X".
\usepackage[ocgcolorlinks,linkcolor=black]{hyperref}
\usepackage{indentfirst} %"Indent" efter section og chapters etc.
\usepackage{graphicx} % for plotting graphs in matrix
\usepackage{subcaption} % figure "footnotes"
\usepackage[font=small,compatibility=false]{caption}
\usepackage[nottoc,numbib]{tocbibind}    

% bibliography
\usepackage{csquotes}
\usepackage{biblatex}
\addbibresource{bibliography.bib}

% Ændring af titler
\usepackage{sectsty}
\subsectionfont{\normalfont\bfseries}
\subsubsectionfont{\itshape}

%Rstudio pakker
\usepackage[svgnames]{xcolor}
\usepackage{listings}

\lstset{language=R,
basicstyle=\scriptsize\ttfamily,
commentstyle=\ttfamily\color{gray},
numbers=left,
numberstyle=\ttfamily\color{gray}\footnotesize,
stepnumber=1,
numbersep=5pt,
backgroundcolor=\color{white},
showspaces=false,
showstringspaces=false,
showtabs=false,
frame=single,
tabsize=2,
captionpos=b,
breaklines=true,
breakatwhitespace=false,
title=\lstname,
escapeinside={},
keywordstyle={},
morekeywords={}
}

% fancy table
\usepackage{booktabs}
\usepackage{multirow}
\pagestyle{fancy}
\fancyhf{}
\fancyhead[L]{\leftmark}
\fancyhead[R]{\rightmark}
\lfoot{\author}
\rfoot{\thepage}
\renewcommand{\headrulewidth}{0.4pt}
\renewcommand{\footrulewidth}{0.4pt}

%symbolforkortelser
\newcommand{\lll}{\mathcal{L}}
\newcommand{\LL}{\Leftrightarrow}
\newcommand{\lp}{\left(}
\newcommand{\rp}{\right)}
\newcommand{\rb}{\right]}
\newcommand{\lb}{\left[}
\newcommand{\lc}{\left\{}
\newcommand{\rc}{\right\}}
\newcommand{\dd}{\mathrm{d}}
\newcommand{\cc}{\mathbb{C}}
\newcommand{\aaa}{\mathbf{A}}
\newcommand{\ee}{\mathbf{E}}
\newcommand{\ff}{\mathcal{F}}
\newcommand{\ggg}{\mathcal{G}}
\newcommand{\hh}{\mathcal{H}}
\newcommand{\pp}{\mathcal{P}}
\newcommand{\qq}{\mathcal{Q}}
\newcommand{\vv}{\mathbf{V}}
\newcommand{\rr}{\mathbf{R}}
\newcommand{\nn}{\mathbf{N}}
\newcommand{\zz}{\mathbf{Z}}
\newcommand{\yy}{\mathbf{Y}}
\newcommand{\nnn}{\mathcal{N}}
\newcommand{\ii}{\mathbf{1}}
\newcommand{\bs}{\text{BS}}
\newcommand{\sumn}{\sum_{i=1}^n}
\renewcommand{\theequation}{1.\arabic{equation}}
\DeclareMathOperator*{\argmin}{\arg\,\min}
\newcommand{\RNum}[1]{\uppercase\expandafter{\romannumeral #1\relax}}

\selectlanguage{english}

\lstset{language=R}

% Environment for theorems etc.
\newtheorem{theorem}{Theorem}[section]
\newtheorem{corollary}{Corollary}[theorem]
\newtheorem{lemma}[theorem]{Lemma}
\newtheorem{assumption}{Assumption}
\newtheorem{proof}{Proof}
\newtheorem{mydef}{Definition}

% Colour citation links (otherwise they are luminescent green)
\hypersetup{colorlinks,linkcolor={black},citecolor={black},urlcolor={black}} 

% TOC depth
\setcounter{tocdepth}{2}

\title{Master's Thesis}
\author{William Gram (khj784) \& Kristian Strand (lpq119)}
\date{\today}

\begin{document}

\pagenumbering{gobble}
\maketitle

\newpage

\pagenumbering{roman}
\rfoot{\thepage}

\tableofcontents

\newpage

\pagenumbering{arabic}
\setcounter{page}{1}

\section{Introduction}

\section{Theory}
\subsection{\textit{The Model}}
\noindent We follow the framework of Guidolin \& Timmermann (2007) \cite{2007} and deploy a general Hidden Markov Model (HMM), which allow for different regimes as well as predictability from other variables such as the lagged return or the dividend yield. For A different asset classes the excess returns on those asset classes, $\rr_t = (R_t^1,R_t^2,...,R_t^A)$, follows an auto regressive (AR) process, with parameters that can vary across \textit{k} regimes driven by the latent variable $S_t$. Furthermore a set of predictor variables, $\zz_t$, is included such that $Y_t=(\rr_t,\zz_t)'$. The predictor variables can potentially be non-linear and depend on the state or linear and independent on states. The model thus becomes:
\begin{align} \label{model}
    \yy_t = \mu_{S_t}+\sum_{j=1}^{p} \aaa_{j,S_t}\yy_{t-j}+\epsilon_t.
\end{align}
$S_t$ takes integer values between 1 to \textit{k} and $\mu_{S_t}$ is the intercept in state $S_t$. $\aaa_{j,S_t}$ is a $A \times A$ matrix of auto regressive coefficients at lag $j$ in state $S_t$. Finally,  $\epsilon_t \sim N(0,\Sigma_{S_t})$ is a vector of return innovations with zero mean and state-specific covariance matrix $\Sigma_{S_t}$, which is assumed to be normally distributed. In the special case of $k=1$ the model is a simple linear auto regressive model with predictor variables and $p$ lags.

$S_t$ is a Markov Chain and the distribution of the one-step transition is given by the transition matrix:
\begin{align}\label{transprob}
 P=\begin{bmatrix}
    p_{11} & p_{21}  & \dots  & p_{N1} \\
    p_{12} & p_{22}  & \dots  & p_{N2} \\
    \vdots & \vdots  & \ddots & \vdots \\
    p_{1N} & p_{2N}  & \dots  & p_{NN}
\end{bmatrix},
\end{align}
where $p_{ij}=P(s_t=h \vert s_{t-1}=i)$. Note that the columns in P must sum to one, i.e. $p_{11}+...+p_{1N}=1$.  The parameters to be estimated is thus:
\begin{align}\label{paramest}
\theta=(\mu_1,...,\mu_N,{\Sigma_1},...,{\Sigma_N}, \aaa_{1,1},...,\aaa_{p,N}, p_{11},...,p_{NN})  
\end{align}
Estimation of the model (\ref{model})-(\ref{transprob}) is complicated by the fact that the switching variable $S_t$ is unobserved, making it difficult to evaluate the likelihood function. As a way to overcome this, we will introduce the expectation-maximisation (EM) algorithm as well as the "forward-backward" algorithm.    

\subsection{\textit{Model Selection}}
\noindent In the following, we will present the theory used to perform a detailed specification analysis, to determine statistical support for our model for the joint distribution of the various asset classes considered. We will consider a range of possible states and lag-orders to choose the most well specified model for our analysis. At first, we will ignore predictor variables.  



\section{Introduction}

\section{Introduction}


\section{Bibliography}
\printbibliography[heading=none]

\end{document}